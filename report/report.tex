\documentclass[10pt]{article}

\usepackage{url}
\usepackage{graphicx}
\usepackage{color}
\usepackage{amsmath}
\usepackage{hyperref}
\usepackage{fullpage}
\usepackage{natbib}

% Code for changing the color of the hyper-references
\hypersetup{
    colorlinks,
    citecolor=blue,
    filecolor=black,
    linkcolor=blue,
    urlcolor=blue
}

\title{Project Waste}
\author{Avikalp Kumar Gupta}
\date{\today}

\begin{document}
\maketitle

\begin{abstract}



\end{abstract}

\section{Problem Statement}

We want to solve the problem of responsible waste disposal for the cities of India. We want Indian cities to look clean and keep its citizens healthy and productive for the betterment of the country. Our aim is to come up with a sustainable solution for the same. The scope of this particular project is currently restricted to the city called \emph{Bengaluru}, capital city of the Karnataka state. 

\section{Introduction}

In India, people are too particular about cleanliness. Their clothes, their houses, their food, everything must be clean. But they make efforts only at micro-levels. In the discussion of solid waste disposal, the job of Indians ends as soon as the garbage is outside the door of their house. They do not realize that the waste that is lying right outside their house still makes their house look filthy and will still make their children sick. 

Human activities generates all forms of waste - solid, liquid (effluents) and gaseous (pollution). In this project, we are focussing the solid waste generated by human activities. This solid waste can be categories in 3 broad categories - dry waste, wet waste and hazardous waste. Wet waste is the more organic readily bio-degradable waste, which mainly consists of the by-products of food preparation and the left-over food.

The BBMP has defined 6 stages of solving the problem of waste management \citep{BBMP:SWMOverview}:
\begin{enumerate}
	\item Collection of waste
	\item Segregation of Waste
	\item Intra-city activies (like sweeping the roads etc.)
	\item Storage of waste
	\item Transportation of the waste
	\item Processing of waste
\end{enumerate}
Although $6^{th}$ stage of this pipeline is the most crucial in terms of responsible solid waste management, our focus will be on the first 5 stages (except maybe the $3^{rd}$ stage). We hope to be able to partner with some companies working on the processing of different kinds of wastes to manage our collections.

Bulk waste generators are defined by BBMP as \emph{``Bulk Generators includes domestic generators-apartments complexes in more than 50 units and institutional and commercial bulk generators who produce more than 10 kg of Municipal Solid Waste"}. Among the bulk waste generators in the city are 15,000 apartment complexes. Any establishment which generates more than 100 kg of wet waste per day is defined as a bulk waste generator. Apartment complexes with more than 50 flats have been identified as bulk generators. Other bulk generators include restaurants, hotels, and choultries\citep{BangaloreMirror:bulk_waste_generators}. \href{http://bbmp.gov.in/documents/10180/2201631/Approved+Vendor+List+with+note+for+Website+21-06-2016.pdf/34567fdf-8d69-4b12-8440-9475ba4a0567}{This document on BBMP's website} lists the various options available to thse bulk waste generators to handle those huge quantities of waste that they produce.

But for the small waste generators, like the individual households, it very costly and difficult to dispose their waste responsibly. The BBMP does collection of the garbage from a lot of them and form the streets as well, but probably this waste is not processed. It probably is openly dumped into nearby rural areas, where is degrades the soil and troubles the life of the residents. At least before December 2014, a lot of Bengaluru's garbage was being dumped at Mandur \citep{BangaloreMirror:bulk_waste_generators}.
\textcolor{red}{We still have to verify whether this is still happening, or has the situation changed. We have some experiments in mind for the same.}

\section{Status Quo}



\subsection{BBMP: Bruhat Bengaluru Mahanagara Palike}

The Municipal Corporation of Bengaluru, i.e. the BBMP, has been active in the last decade in solving the waste management problem of the city. \cite{BBMP:SWMOverview} gives a good overview of the arrangements that the BBMP has planned for managing the city's waste.

\subsubsection{Wet Waste}

\subsubsection{Dry Waste}

\subsection{Hasirudala}

\subsection{Saahas Zero Waste}

\subsection{Daily Dump}

\subsection{Encashea}

\subsection{The Ugly Indian}

\section{Related Work}

\subsection {Paperman}

\subsection{Litterati}

\subsection{Sweden's case}

\subsection{Waste treatment in USA}



\bibliographystyle{plainnat}
\bibliography{report_bib.bib}

\end{document}